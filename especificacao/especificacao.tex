\documentclass[11pt]{article}
\usepackage{sbc-template}
\usepackage[english]{babel}
\usepackage[utf8]{inputenc}
\usepackage{listings}
\usepackage{enumitem}

\lstset{ %
basicstyle=\footnotesize,       % the size of the fonts that are used for the code
numbers=left,                   % where to put the line-numbers
numberstyle=\footnotesize,      % the size of the fonts that are used for the line-numbers
stepnumber=1,                   % the step between two line-numbers. If it is 1 each line will be numbered
numbersep=7pt,                   % choose the background color. You must add \usepackage{color}
showspaces=false,               % show spaces adding particular underscores
showstringspaces=false,         % underline spaces within strings
showtabs=false,                 % show tabs within strings adding particular underscores
frame=single,           % adds a frame around the code
tabsize=2,          % sets default tabsize to 2 spaces
breaklines=true,        % sets automatic line breaking
breakatwhitespace=false,    % sets if automatic breaks should only happen at whitespace
escapeinside={\%*}{*)},          % if you want to add a comment within your code
extendedchars=true,
literate={á}{{\'a}}1 {ã}{{\~a}}1 {é}{{\'e}}1
}

\title{Especificação da linguagem}

\author{Lucas Amaral $^1$}

\date{\today}

\address{Departamento de Ciência da Computação - Universidade de Brasília}

\pagenumbering{roman}
\begin{document}
\maketitle


\section{Descrição do trabalho}
Este trabalho consiste na implementação de um tradutor para uma linguagem destinada à 
área de programação funcional.
A linguagem será baseada em outras, como ML e Haskell:
pura, com funções de primeira ordem e listas como tipo primitivo.
O diferencial desta linguagem é a possibilidade de declarações de funções de uma maneira procedural,
porém sem uso de variáveis globais ou modificação de estado fora de seu escopo direto.

Em linguagens puramente funcionais, as funções são definidas a partir de composições, descrevendo 
seu comportamento através do relacionamento entre as funções que a compõem,
ao invés de uma descrição dos passos a serem realizados, como em linguagens mais tradicionais e
estruturadas, como C.
Este alto nível de abstração, acompanhado da rigidez em relação a efeitos colaterais e 
de um sistema de tipos forte,
provê qualidades há muito conhecidas no campo de estudo de linguagens de programação, 
como diminuição do acoplamento do código, 
facilitamento no raciocínio sobre o funcionamento de funções (estas não modificam o meio) 
e também auxílio no estudo de propriedades lógicas e matemáticas das funções, 
devido ao embasamento teórico do Cálculo Lâmbda e da Teoria de Tipos.

Porém, nem sempre a maneira funcional é a mais intuitiva para a concretização de um algoritmo.
Aprendemos a projetar nossas soluções para problemas como uma sequência de passos,
o que pode não ter uma descrição simples através de composição de funções. 
Assim, uma modificação à esta estrutura é proposta, permitindo que funções sejam, também, declaradas
imperativamente. Mas, diferentemente das linguagens estruturadas comuns, 
a modificação do ambiente não é permitida, mantendo, assim, a "pureza" da função,
sem perder as vantagens mencionadas anteriormente.

Essencial para o bom funcionamento do compilador é um algoritmo básico para inferência de tipos.
Apesar de já existirem algoritmos bem estabelecidos com este propósito 
para linguagens funcionais, modificações em sua estrutura podem vir a serem necessárias devido 
à adição das definições imperativas.


\section{ Explicação semântica da linguagem }

Abaixo encontram-se dois exemplos que demonstram os aspectos semânticos da linguagem, acompanhados de uma breve explicação.


\subsection{Primeiro exemplo}

\begin{lstlisting}[basicstyle=\small]
incListBy :: Integer -> [Integer] -> [Integer];
incListBy 0 xs = xs;
incListBy _ [] = [];
incListBy i xs = do {
    ys <- [];
    while( xs /= [] ) {
        x:xs <- xs;
        ys <- (x + i):ys;
    };
    rev ys;
} where {
    rev [] = [];
    rev x:xs = (rev xs) ++ [x];
}
\end{lstlisting}


Pontos a se observar:

\begin{itemize}[leftmargin=.5in]
  \item Declarações de tipo
  \item Declaração de função com pattern matching
  \item Pattern matching em atribuições
  \item Declaração de função limitada a um escopo
  \item Funções definidas imperativamente
\end{itemize}

Nas primeiras linhas, observamos uma estrutura semelhante à de outras linguagens funcionais,
como Haskell.
Inicia-se uma função com uma declaração de seu tipo, contendo tipos dos argumentos e de retorno.
A função pode ser definida para alguns valores específicos, com Pattern Matching. 
A ordem das declarações faz diferença, seguindo uma abordagem top-down. 
Também são usados wildscores ``\_" e variáveis.

A partir deste ponto, já notamos uma grande diferença: a descrição imperativa da função. 
Nela, podemos notar o uso de estruturas de repetição, atualização do valor
de variáveis e, novamente, o uso de Pattern Matching. Podemos notar também que a função é avaliada 
para o valor da última linha executada. No caso \emph{rev ys}.
A limitação de escopo também está presente, explicitada pelo uso da palavra-chave \emph{where}. 
Nela, a função \emph{rev} é declarada, mas tem seu escopo limitado à função. 


\subsection{Segundo exemplo}

\begin{lstlisting}[basicstyle=\small]
sumBelow10 :: [Integer] -> Integer;
sumBelow10 [] = 0;
sumBelow10 xs = do {
    sum <- 0;
    while(xs /= []) {
        x:xs <- xs;
        aux  <- if (x < 10) then {
                    yield x;
                } else {
                    yield 0;
                };
        sum <- sum + aux;
    };
    sum;
}
\end{lstlisting}

Pontos a observar:
\begin{itemize}[leftmargin=.5in]
\item Expressões possuem valores
\item Função declarada imperativamente retorna o último valor computado.
\end{itemize}


Aqui, vemos que o valor da expressão \emph{if-then-else} está sendo atribuída à uma variável 
\emph{aux}, diferente da expressão while, que não tem seu valor atribuído a nada.
Isto se deve ao uso da palavra-chave \emph{yield}, que faz com que estas estruturas tenham retornem 
um valor no contexto imperativo. No caso do \emph{if}, ambas as branches precisam
de um \emph{yield}, e o tipo deve ser compatível.
Caso o \emph{yield} não seja usado, como no caso do \emph{while}, a expressão retorna o valor 
unitário, "()".

Vale ressaltar que o \emph{yield} não funciona da mesma maneira que o \emph{return} das linguagens 
tradicionais, pois não pode ser usado para terminar a execução no meio de um bloco.
Diferentemente, ele é opcional, e só pode ser usado como última instrução de uma estrutura de 
loop/condição.
Vemos também que a função retorna o valor de \emph{sum}, pois é a última linha a ser executada.


O uso de estruturas imperativas e modificação de variáveis 
internas não afeta a 
\emph{previsibilidade} da função. Para os mesmo valores de 
entrada, ela sempre retorna o mesmo 
resultado.

\section{Gramática da linguagem}

Abaixo encontra-se a gramática para a linguagem. Alguns detalhes importantes:
\begin{itemize}[leftmargin=.5in]
\item O uso de `;' para delimitar extensão das declarações de
função e statments no caso imperativo. 
\item O uso das variáveis 
do tipo \emph{op\_precN}. Nelas, \emph{N} representa o nível de 
precedência do operador, em ordem crescente. 
\item Notar também a variável \emph{procdecl}, referente à declaração procedural de função.
\end{itemize}

\begin{lstlisting}[basicstyle=\small]
program:  
    line_elems

line_elems: 
    line_elem line_elems
    | line_elem

line_elem:
    fundecl
    | procdecl
    | funtype_decl

fundecl: 
    ID args '=' expr ';'
    | ID args '=' expr where_exp 
    | ID '=' expr ';'
    | ID '=' expr where_exp

args:
    arg_value args 
    | arg_value
    | WILDSCORE

arg_value:
    list_value
    | basic_value
    | '(' arg_value ')'

basic_value: 
    INT
    | FLOAT
    | TRUE
    | FALSE
    | ID
    | `(' `)'

list_value:
    arg_value ':' list_value
    | built_list_value

built_list_value:
    '[' ']'
    | '[' args ']'

funtype_decl:
    ID '::' funtype ';'

funtype:
    basic_type
    | '(' funtype ')'
    | basic_type '->' funtype

basic_type:
    INTEGER
    | FLOAT_T
    | BOOL
    | '[' types ']' 
    | ID
    | `(' `)'

types:
    basic_type
    | basic_type ',' types

expr:
    op_prec1
    | appexp
    | ifexpr
    | yieldexpr

ifexpr:
    'if' expr 'then' '{' expr '}' 'else' '{' expr '}'

yieldexpr:
    "yield" ifexpr
    | "yield" appexpr
    | "yield" op_prec1


op_prec1:
     op_prec2 '||' op_prec1
     | op_prec2

op_prec2:
     op_prec3 '&&' op_prec2
     | op_prec3

op_prec3:
     op_prec3 '==' op_prec4
     | op_prec3 '/=' op_prec4
     | op_prec3 '<' op_prec4
     | op_prec3 '<=' op_prec4
     | op_prec3 '>' op_prec4
     | op_prec3 '>=' op_prec4
     op_prec4

op_prec4:
     op_prec5 ':' op_prec4
     | op_prec5 '++' op_prec4
     | op_prec5

op_prec5:
    op_prec5 '+' op_prec6
    | op_prec5 '-' op_prec6
    | op_prec6

op_prec6:
     op_prec6 '%' op_prec7
     | op_prec7

op_prec7:
    op_prec7 '*' op_prec8
    | op_prec7 '/' op_prec8
    | op_prec8

op_prec8:
    basic_value
    | list_expr
    | '(' expr ')'
    | '-' expr

list_expr:
    | '[' exprs ']'
    | '[' ']'
    
exprs:
    expr
    | expr ',' exprs

appexp:
    ID expr

where_exp:
    'where' '{' line_elems '}' 

procdecl:
    ID args '=' 'do' '{' stmts '}'
    | ID args '=' 'do' '{' stmts '}' where_exp
    | ID '=' 'do' '{' stmts '}'
    | ID '=' 'do' '{' stmts '}' where_exp

stmts:
    stmt
    | stmt stmts

stmt:
    basic_value '<-' expr ';'
    | basic_value '<-' while_expr ';'
    | expr ';'
    | while_expr ';'

while_expr:
    'while' '(' expr ')' '{' stmts '}' 
\end{lstlisting}

\section{Referências Bibliográficas}
[1] The Haskell 98 Report - https://www.haskell.org/onlinereport/index.html $^1$

$^1$ Utilizado como referência para estruturas da linguagem Haskell e predecências de operadores.

\end{document}
