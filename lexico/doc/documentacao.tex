\documentclass[11pt]{article}
\usepackage{sbc-template}
\usepackage[english]{babel}
\usepackage[utf8]{inputenc}
\usepackage{listings}
\usepackage{enumitem}

\lstset{ %
basicstyle=\footnotesize,       % the size of the fonts that are used for the code
numbers=left,                   % where to put the line-numbers
numberstyle=\footnotesize,      % the size of the fonts that are used for the line-numbers
stepnumber=1,                   % the step between two line-numbers. If it is 1 each line will be numbered
numbersep=7pt,                   % choose the background color. You must add \usepackage{color}
showspaces=false,               % show spaces adding particular underscores
showstringspaces=false,         % underline spaces within strings
showtabs=false,                 % show tabs within strings adding particular underscores
frame=single,           % adds a frame around the code
tabsize=2,          % sets default tabsize to 2 spaces
breaklines=true,        % sets automatic line breaking
breakatwhitespace=false,    % sets if automatic breaks should only happen at whitespace
escapeinside={\%*}{*)},          % if you want to add a comment within your code
extendedchars=true,
literate={á}{{\'a}}1 {ã}{{\~a}}1 {é}{{\'e}}1
}

\title{Documentação do analisador léxico}

\author{Lucas Amaral $^1$}

\address{Departamento de Ciência da Computação - Universidade de Brasília}

\begin{document}
\maketitle


\section{Descrição do trabalho}

Este trabalho consiste na implementação do analisador léxico utilizando a ferramenta Flex.
Uma pequena modificação da gramática foi feita em relação à anteriormente apresentada. 
Mais especificamente, as diretrizes de entrada e saída foram adicionadas, 
como vai ser exibido posteriormente.

\section{Visão geral do projeto}

Foi utilizada a linguagem C++ para o programa, principalmente pelo fornecimento de estruturas
de dados prontas em suas bibliotecas. 

O programa irá receber, por linha de comando, o arquivo a ser escaneado. Cada token será
lido e terá seu nome correspondente impresso na tela. 
Quebras de linha no arquivo fonte são replicadas na saída. 
Caso algum erro ocorra, este será guardado e só será notificado no final da 
execução do programa. É utilizada uma estrutura em uma lista para armazenamento dos erros.

A função \emph{handle\_token(int token)} é utilizada para fazer o tratamento de um token lido,
imprimindo na tela os adequados e adicionando à lista de erros caso algo de incorreto seja
encontrado. Ela consiste, basicamente, de um vetor contendo os tokens como strings, e \emph{if-then-else} com um tratamento
apropriado para cada token. Quando um token não identificado é lido, o código seguinte é 
executado. Nele \emph{comp\_error\_t} é uma estrutura que representa um erro, contendo um tipo de
erro e uma mensagem a ser mostrada. A lista de erros é chamada de \emph{error\_list}. A linha
e coluna onde o token inválido é encontrado são adicionadas no início da mensagem.


\begin{lstlisting}[basicstyle=\small]
    std::string aux = "\n" ;
    aux += to_string(line);
    aux += ":";
    aux += to_string(col);
    aux += " Token não reconhecido \"";
    aux += yytext;
    aux += "\"";
    error_list.push_back(comp_error_t(0, aux));
\end{lstlisting}

Este tratamento foi escolhido por dois motivos:

\begin{itemize}[leftmargin=.5in]
  \item Organização da saída do programa
  \item Reaproveitamento da estrutura de tratamento de erros para os próximos trabalhos
\end{itemize}

Vale ressaltar que o único tipo de erro tratado nesta fase é o erro estritamente léxico. 
Apenas tokens inválidos são identificados como um erro por este analisador.

Foi incluida também uma função \emph{main} para realização da do recebimento do nome do
arquivo por linha de comando, chamada à função específica do parser e impressão dos resultados.



\subsection{Segundo exemplo}

\begin{lstlisting}[basicstyle=\small]
sumBelow10 :: [Integer] -> Integer;
sumBelow10 [] = 0;
sumBelow10 xs = do {
    sum <- 0;
    while(xs /= []) {
        x:xs <- xs;
        aux  <- if (x < 10) then {
                    yield x;
                } else {
                    yield 0;
                };
        sum <- sum + aux;
    };
    sum;
}
\end{lstlisting}

Pontos a observar:
\begin{itemize}[leftmargin=.5in]
\item Expressões possuem valores
\item Função declarada imperativamente retorna o último valor computado.
\end{itemize}


Aqui, vemos que o valor da expressão \emph{if-then-else} está sendo atribuída à uma variável 
\emph{aux}, diferente da expressão while, que não tem seu valor atribuído a nada.
Isto se deve ao uso da palavra-chave \emph{yield}, que faz com que estas estruturas tenham retornem 
um valor no contexto imperativo. No caso do \emph{if}, ambas as branches precisam
de um \emph{yield}, e o tipo deve ser compatível.
Caso o \emph{yield} não seja usado, como no caso do \emph{while}, a expressão retorna o valor 
unitário, "()".

Vale ressaltar que o \emph{yield} não funciona da mesma maneira que o \emph{return} das linguagens 
tradicionais, pois não pode ser usado para terminar a execução no meio de um bloco.
Diferentemente, ele é opcional, e só pode ser usado como última instrução de uma estrutura de 
loop/condição.
Vemos também que a função retorna o valor de \emph{sum}, pois é a última linha a ser executada.


O uso de estruturas imperativas e modificação de variáveis 
internas não afeta a 
\emph{previsibilidade} da função. Para os mesmo valores de 
entrada, ela sempre retorna o mesmo 
resultado.

\section{Gramática da linguagem}

Abaixo encontra-se a gramática para a linguagem. Alguns detalhes importantes:
\begin{itemize}[leftmargin=.5in]
\item O uso de `;' para delimitar extensão das declarações de
função e statments no caso imperativo. 
\item O uso das variáveis 
do tipo \emph{op\_precN}. Nelas, \emph{N} representa o nível de 
precedência do operador, em ordem crescente. 
\item Notar também a variável \emph{procdecl}, referente à declaração procedural de função.
\end{itemize}

\begin{lstlisting}[basicstyle=\small]
program:  
    line_elems

line_elems: 
    line_elem line_elems
    | line_elem

line_elem:
    fundecl
    | procdecl
    | funtype_decl

fundecl: 
    ID args '=' expr ';'
    | ID args '=' expr where_exp 
    | ID '=' expr ';'
    | ID '=' expr where_exp

args:
    arg_value args 
    | arg_value
    | WILDSCORE

arg_value:
    list_value
    | basic_value
    | '(' arg_value ')'

basic_value: 
    INT
    | FLOAT
    | TRUE
    | FALSE
    | ID
    | `(' `)'

list_value:
    arg_value ':' list_value
    | built_list_value

built_list_value:
    '[' ']'
    | '[' args ']'

funtype_decl:
    ID '::' funtype ';'

funtype:
    basic_type
    | '(' funtype ')'
    | basic_type '->' funtype

basic_type:
    INTEGER
    | FLOAT_T
    | BOOL
    | '[' types ']' 
    | ID
    | `(' `)'

types:
    basic_type
    | basic_type ',' types

expr:
    op_prec1
    | appexp
    | ifexpr
    | yieldexpr

ifexpr:
    'if' expr 'then' '{' expr '}' 'else' '{' expr '}'

yieldexpr:
    "yield" ifexpr
    | "yield" appexpr
    | "yield" op_prec1


op_prec1:
     op_prec2 '||' op_prec1
     | op_prec2

op_prec2:
     op_prec3 '&&' op_prec2
     | op_prec3

op_prec3:
     op_prec3 '==' op_prec4
     | op_prec3 '/=' op_prec4
     | op_prec3 '<' op_prec4
     | op_prec3 '<=' op_prec4
     | op_prec3 '>' op_prec4
     | op_prec3 '>=' op_prec4
     op_prec4

op_prec4:
     op_prec5 ':' op_prec4
     | op_prec5 '++' op_prec4
     | op_prec5

op_prec5:
    op_prec5 '+' op_prec6
    | op_prec5 '-' op_prec6
    | op_prec6

op_prec6:
     op_prec6 '%' op_prec7
     | op_prec7

op_prec7:
    op_prec7 '*' op_prec8
    | op_prec7 '/' op_prec8
    | op_prec8

op_prec8:
    basic_value
    | list_expr
    | '(' expr ')'
    | '-' expr

list_expr:
    | '[' exprs ']'
    | '[' ']'
    
exprs:
    expr
    | expr ',' exprs

appexp:
    ID expr

where_exp:
    'where' '{' line_elems '}' 

procdecl:
    ID args '=' 'do' '{' stmts '}'
    | ID args '=' 'do' '{' stmts '}' where_exp
    | ID '=' 'do' '{' stmts '}'
    | ID '=' 'do' '{' stmts '}' where_exp

stmts:
    stmt
    | stmt stmts

stmt:
    basic_value '<-' expr ';'
    | basic_value '<-' while_expr ';'
    | expr ';'
    | while_expr ';'

while_expr:
    'while' '(' expr ')' '{' stmts '}' 
\end{lstlisting}

\section{Referências Bibliográficas}
[1] The Haskell 98 Report - https://www.haskell.org/onlinereport/index.html $^1$

$^1$ Utilizado como referência para estruturas da linguagem Haskell e predecências de operadores.

\end{document}
