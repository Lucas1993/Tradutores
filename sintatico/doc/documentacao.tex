\documentclass[11pt]{article}
\usepackage{sbc-template}
\usepackage[portuguese]{babel}
\usepackage[utf8]{inputenc}
\usepackage{listings}
\usepackage{enumitem}
\usepackage{indentfirst}

\lstset{ %
basicstyle=\footnotesize,       % the size of the fonts that are used for the code
numbers=left,                   % where to put the line-numbers
numberstyle=\footnotesize,      % the size of the fonts that are used for the line-numbers
stepnumber=1,                   % the step between two line-numbers. If it is 1 each line will be numbered
numbersep=7pt,                   % choose the background color. You must add \usepackage{color}
showspaces=false,               % show spaces adding particular underscores
showstringspaces=false,         % underline spaces within strings
showtabs=false,                 % show tabs within strings adding particular underscores
frame=single,           % adds a frame around the code
tabsize=2,          % sets default tabsize to 2 spaces
breaklines=true,        % sets automatic line breaking
breakatwhitespace=false,    % sets if automatic breaks should only happen at whitespace
escapeinside={\%*}{*)},          % if you want to add a comment within your code
extendedchars=true,
literate={á}{{\'a}}1 {ã}{{\~a}}1 {é}{{\'e}}1
}

\title{Documentação do analisador sintático}

\author{Lucas Amaral $^1$}

\address{Departamento de Ciência da Computação - Universidade de Brasília}

\begin{document}
\maketitle


\section{Descrição do trabalho}

Este trabalho consiste na implementação do analisador sintático utilizando a ferramenta Bison e reutilizando o analisador léxico do trabalho anterior.
Diversas alterações foram feitas na gramática, em especial na regra de appexpr e no tipo de argumentos recebidos por funções.

\section{Analisador Sintático}

\subsection{Visão geral do projeto}

O programa irá receber, por linha de comando, o arquivo a ser analisado. 
Em caso de entrada pertencente à linguagem, será exibida uma árvore sintática referente ao código,
além da tabela de símbolos, com os identificadores e os tipos já declarados.
Em caso de erro, a árvore e a tabela não serão impressos, sendo exibido um relatório de erros em seu lugar.

\subsection{Tratamento de erros}

O tratamento de erros é dividido em duas partes, uma no analisador sintático e outra no léxico.
Quando um erro é detectado em qualquer uma das partes, uma flag de erro é ativada para impedir
a impressão da árvore e tabela de símbolos.

No analisador léxico, quando um erro ocorre, a função \emph{lex\_error} é chamada, adicionando o erro na lista de erros,
com a linha, coluna e mensagem a ser exibida. Quando um identificador inválido é encontrado, o analisador léxico
o envia para o sintático mesmo assim, para tentar encontrar mais erros antes de terminar a execução.

No analisador sintático, a função que registra erros é a \emph{yyerror}. Esta também irá registrar a linha e coluna
onde o erro ocorreu, e irá utilizar como mensagem o erro passado pelo Bison.
Para tentar evitar terminação imediata ao encontrar um erro, a palavra reservada \emph{error} é usada em algumas regras,
tratando o erro localmente. Isto é feito particularmente nas regras das variáveis \emph{fundecl} e \emph{stmt}.   


Para registrar os erros, as funções \emph{make\_error} e \emph{add\_error} são usadas, sendo implementadas em um arquivo separado, chamado
\emph{list\_error}. A primeira aloca uma struct do tipo comp\_error\_t e inicializa com os parâmetros, retornando o ponteiro,
e a segunda adiciona um erro em uma lista dada no primeiro argumento.

Este tratamento foi escolhido por vários motivos:

\begin{itemize}[leftmargin=.5in]
  \item Organização da saída do programa
  \item Reaproveitamento da estrutura de tratamento de erros para os próximos trabalhos
  \item Tentar tratar o máximo de erros possíveis em uma execução
\end{itemize}

Foi incluida também uma função \emph{main} para realização da do recebimento do nome do
arquivo por linha de comando, chamada à função específica do parser e impressão dos resultados.


\section{Sobre a gramática da linguagem}

Foram feitas diversas mudanças na gramática. A grande maioria delas envolve pequenos 
detalhes que serão omitidos a fim de diminuir a extensão deste documento. 
Serão mencionadas apenas as mudanças principais.

\subsection{As regras da variável appexpr}

Estas regras foram modificadas a fim de eliminar os erros de reduce/reduce.  
Foi criado uma nova variável para limitar as substituições de appexpr.
A mudança foi feita como demonstrado abaixo.

\begin{lstlisting}[basicstyle=\small]
appexpr:
       appexpr nonapp 
       | nonapp nonapp 

nonapp:
      basic_value
      | '(' expr ')'
\end{lstlisting}

Isto limita as substituições de ``expr $\rightarrow$ appexpr $\rightarrow$ expr", que causavam conflitos.

\subsection{A variável yieldexpr}

Foi retirada a derivação de expr para yieldexpr. 
A regra que gera yieldexpr agora é a stmt $\rightarrow$ yieldexpr. 

Esta mudança acarreta numa grande diferença, não só na sintaxe,
como também na semântica da linguagem. 
Agora, os procedimentos não retornam um valor explicitamente, sendo necessário um yield
explícito, que termina o procedimento e retorna um valor.
A estrutura \emph{while} passa a precisar também de um yield, mas o \emph{if-then-else}
continua funcionando sem, normalmente.

\subsection{A variável id\_list}

Nas regras de stmt, a atribuição era feita com basic\_value no lado
esquerdo, mas foi mudada para a variável id\_list, que consiste de um ou mais
identificadores separados por `:'. Isto foi feito
a fim de garantir consistência, já que não faz sentido
atribuir um valor a outro, além de possibilidar casamento de padrões
em atribuições de listas.

A variável id\_list também substitui a derivação de ID em arg\_value. Ou seja,
agora, listas de identificadores podem ser usados como argumentos,
também para permitir o casamento de padrões.

\pagebreak

\section{Construção da árvore sintática}

A árvore sintática é formada de várias structs diferentes, uma para cada variável da gramática.
São usadas unions internas para representar cada possibilidade de derivação. Um campo de um tipo enum é usado para identificar
qual derivação foi feita.

Veja o exemplo abaixo.

\begin{lstlisting}[basicstyle=\small]
struct line_t {
    enum { L_FUN, L_PROC, L_FUNTYPE } opt_type;
    union {
        fun_t* fun;
        proc_t* proc;
        funtype_decl_t* funtype;
    } opt;
}; 
\end{lstlisting}

A árvore é construída utilizando as ações realizadas nas reduções da gramática. 
Ou seja, a construção da árvore está concluída quando a redução da regra program $\rightarrow$ line\_elems é feita.
Neste momento, a função que imprime a árvore é chamada no nó raíz. 

\begin{lstlisting}[basicstyle=\small] 
line_elem:
    fundecl {
        line_t* tmp = NEW(line_t);
        tmp->opt_type = L_FUN;
        tmp->opt.fun = $1;
        $$ = tmp; 
    }
\end{lstlisting}

A macro \emph{NEW} aloca e zera o espaço de memória. Cada struct tem um campo para dizer qual a derivação feita,
além de um ponteiro para o filho.

As diretivas \emph{\%type} e \emph{\%token} são utilizadas para tornar o código mais sucinto.

\subsection{A função de impressão de árvore}

Esta função se tornou inevitavelmente extensa, devido à estrutura heterogênea da árvore.
A função recebe 3 parâmetros 
\begin{itemize}
    \item Um nó do tipo YYSTYPE, a union do próprio bison. Esta serve para envolver o nó real da árvore,
        permitindo que apenas uma função trate cada tipo de struct diferente.
    \item Um char \emph{node\_type}. Este char serve como um switch para identificar em que tipo de nó estamos.
    \item Um inteiro \emph{ident}. Ele serve para registrar em que nível a chama atual se encontra, para que
        seja possível imprimir a árvore de maneira visualmente correta.
\end{itemize}

O corpo da função consiste basicamente de um grande switch-case no parâmetro \emph{node\_type}.

Cada caso irá imprimir as informações referente ao nó e chamar recursivamente a função de impressão
para os nós filhos, incrementando o parâmetro \emph{ident} neles.

\section{Tabela de símbolos}

A implementação da tabela de símbolos foi extremamente simples. Foi necessária apenas a implementação de uma
lista, cada nó armazenando o nome do símbolo e o seu tipo. 

Vale lembrar que nesta etapa só é possível identificar o tipo de funções, e somente quando este é explicitamente declarado.
É necessário também que a declaração do tipo venha antes da declaração da função, caso contrário o tipo não é registrado.
Na análise semântica, declarar a função antes do tipo constituirá erro.

Os símbolos são adicionados na tabela quando um identificador qualquer é encontrado.

\begin{lstlisting}[basicstyle=\small]
    funtype_decl:
        ID DOUBLECOLON funtype ';' {
            funtype_decl_t* tmp = NEW(funtype_decl_t);
            tmp->label = $1;
            tmp->type = $3;
            $$ = tmp;
            add_symbol(&symtable, $1,  $3);

        }
	;
\end{lstlisting}


\section{Descrição dos arquivos de teste}

Existem 4 arquivos de teste contidos na pasta \emph{testes}.
\begin{itemize}
    \item O arquivo \emph{correto1.txt} mostra um programa simples da linguagem. Observar os operadores e os identificadores com números.
    \item O arquivo \emph{correto2.txt} contém mais alguns exemplos, como constantes booleanas, comentários e erros sendo ignorados dentro dos comentários, 
        em especial, identificador começando com underscore.
    \item O arquivo \emph{incorreto1.txt} já contém alguns erros. Identificadores começando com underscore e caracteres inválidos estão presentes.
    \item O arquivo \emph{incorreto2.txt} também contém alguns erros. Em especial vale notar que \emph{\#} não inicia um comentário,
        \emph{x - -} não retorna dois tokens, mas três. Identificadores não podem começar com números e nem underscores.
\end{itemize}


\section{Dificuldades encontradas}

Não foram encontradas muitas dificuldades no desenvolvimento do trabalho. A única que vale a pena
ser mencionada é a identificação de erros que o analisador não percebe imediatamente.
Um exemplo é identificar números seguidos de letras como um erro léxico. O analisador léxico retorna dois tokens,
\textless T\_NUM\textgreater e \textless T\_ID\textgreater, ao invés de cair na regra de token não identificado. Para tratar este tipo de erro, uma regra foi
adicionada para cada um.

\section{Referências Bibliográficas}
[1] The Haskell 98 Report - https://www.haskell.org/onlinereport/index.html $^1$

$^1$ Utilizado como referência para estruturas da linguagem Haskell e predecências de operadores.

\section{Anexo: Gramática}

\begin{lstlisting}[basicstyle=\small]
program:  
    line_elems;

line_elems: 
    line_elem line_elems 
    | line_elem 
	;

line_elem:
    fundecl 
    | procdecl 
    | funtype_decl 
	;

fundecl: 
    ID args '=' expr ';' 
    | ID args '=' expr where_exp 
    | ID '=' expr ';' 
    | ID '=' expr where_exp 
	;

args:
    arg_value args  
    | WILDSCORE args  
    | arg_value 
    | WILDSCORE 
	;

arg_value:
    '(' list_value ')' 
    | basic_value 
    | '(' arg_value ')' 
	;

id_list:
       ID ':' id_list 
      | WILDSCORE ':' id_list 
      | ID  
      | WILDSCORE  
      ;

basic_value: 
    NUMBER 
    | FLOATNUM 
    | BOOLVAL 
    | ID 
    | '(' ')' 
	;

list_value:
    basic_value ':' list_value 
    |'[' list_args ']' ':' list_value 
    | WILDSCORE ':' WILDSCORE 
    | WILDSCORE ':' list_value 
    | built_list_value 
    | basic_value ':' ID 
    | basic_value ':' WILDSCORE 
	;

list_args:
         arg_value 
         | arg_value ',' list_args 
         ;

built_list_value:
    '[' ']' 
    | '[' list_args ']' 
	;

funtype_decl:
    ID DOUBLECOLON funtype ';' 
	;

funtype:
    basic_type 
    | '(' funtype ')' 
    | basic_type RARROW funtype 
    | '(' funtype ')' RARROW funtype 
	;

basic_type:
    INTEGER 
    | FLOAT 
    | BOOL 
    | '[' basic_type ']' 
    | ID 
    | '(' ')' 
	;

appexpr:
       appexpr nonapp 
       | nonapp nonapp 
	;

nonapp:
      basic_value 
      | '(' expr ')' 
      ;

expr:
    op_prec1 
    | appexpr 
    | ifexpr 
	;

ifexpr:
    IF expr THEN expr ELSE expr 
	;

yieldexpr:
    YIELD ifexpr 
    | YIELD appexpr 
    | YIELD op_prec1 
	;

op_prec1:
    op_prec2 OR op_prec1 
    | op_prec2 
	;

op_prec2:
    op_prec3 AND op_prec2 
    | op_prec3 
	;

op_prec3:
    op_prec4 EQUALS op_prec3 
    | op_prec4 DIFF op_prec3 
    | op_prec4 '<' op_prec3 
    | op_prec4 LEQ op_prec3 
    | op_prec4 '>' op_prec3 
    | op_prec4 GEQ op_prec3 
    | op_prec4 
	;

op_prec4:
    op_prec5 ':' op_prec4 
    | op_prec5 APPEND op_prec4 
    | op_prec5 
	;

op_prec5:
    op_prec5 '+' op_prec6 
    | op_prec5 '-' op_prec6 
    | op_prec6 
	;

op_prec6:
     op_prec6 '%' op_prec7
     | op_prec7 
	 ;

op_prec7:
    op_prec7 '*' op_prec8
    | op_prec7 '/' op_prec8
    | op_prec8 
	;

op_prec8:
    basic_value 
    | list_expr 
    | '(' expr ')' 
    | '-' op_prec8 
	;

exprs:
    expr 
    | expr ',' exprs 
	;

list_expr:
     '[' exprs ']' 
    | '[' ']' 
	;


where_exp:
    WHERE ''  
	;


procdecl:
    ID args '=' DO '{' stmts '}' 
    | ID args '=' DO '{' stmts '}' where_exp 
    | ID '=' DO '{' stmts '}' 
    | ID '=' DO '{' stmts '}' where_exp  
	;

stmts:
    stmt 
    | stmt stmts 
	;

stmt:
    id_list LARROW expr ';' 
    | id_list LARROW while_expr 
    | id_list LARROW io_stmt ';'
    | yieldexpr ';'
    | while_expr 
    | io_stmt ';' 
	;

io_stmt:
    READINT     
    | READFLOAT 
    | READBOOL 
    | PRINT expr 
	;

while_expr:
    WHILE '(' expr ')' '{' stmts '}' 
	;
\end{lstlisting}


\end{document}
